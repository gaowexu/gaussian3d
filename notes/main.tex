\documentclass[UTF8,a4paper,12pt]{article}
\usepackage{ctex} % 支持中文
\usepackage{setspace} % 设置行距
\usepackage{titlesec} % 设置章节标题格式
\usepackage{geometry} % 设置页面布局
\usepackage{amsmath}
\usepackage{bm} % 加粗数学符号的宏包
\usepackage{amssymb}


% 设置页面布局
\geometry{left=3cm,right=3cm,top=3cm,bottom=3cm}

% 设置行距
\linespread{1.25}

% 设置章节标题格式
\titleformat{\section}{\bfseries\Large}{\thesection}{1em}{}
\titleformat{\subsection}{\bfseries\large}{\thesubsection}{1em}{}
\titleformat{\subsubsection}{\bfseries\normalsize}{\thesubsubsection}{1em}{}

% 标题、作者、作者单位
\title{完全理解《3D Gaussian Splatting for Real-Time Radiance Field Rendering》指南}
\author{徐高伟}
\date{2024-04-11} % 可以留空,不显示日期


\begin{document}

\maketitle


% 正文
\section{一维与三维高斯函数分布}
一个服从均值$\mu$,方差$\sigma$的一维高斯函数,其函数表达式为:

\begin{equation}
f_{\mu, \sigma}(x) = \frac{1}{\sqrt{2\pi} \sigma} e ^ {- \frac{(x- \mu)^2}{2 \sigma ^2}}
\end{equation}



三维高斯函数的函数表达式为:
\begin{equation}
f_{\bm{\mu}, \bm{\Sigma}} = \left(  \frac{1}{ \sqrt{2\pi}^3 det(\bm{\Sigma})} \right) e ^ {-\frac{1}{2}(\bm{x} - \bm{\mu})^T \bm{\Sigma} ^ {-1} (\bm{x} - \bm{\mu})}  
\end{equation}


其中 $\bm{x} = (x, y, z)^T \in \mathbb{R}^{3 \times 1}$ 是三维坐标向量,它是一个列向量,$\bm{ \mu } = (\mu_x, \mu_y, \mu_z) ^T \in \mathbb{R}^{3 \times 1}$ 是椭球中心,控制三维高斯椭球在世界坐标系空间中的位置。协方差矩阵$\bm{\Sigma}$控制三维高斯椭球在三个轴方向的伸缩和旋转,其中协方差矩阵的特征向量就是椭球对称轴。协方差矩阵$\bm{\Sigma}$的定位如下:

\begin{equation}
\bm{\Sigma} = \begin{pmatrix}
\sigma_x ^2 					& cov(x, y) 					& 	cov(x, z)				\\
cov(y, x) 						& \sigma_y ^2 					&	 cov(y, z)			 	\\
cov(z, x) 						& cov(z, y) 					& 	\sigma_z ^2 			
\end{pmatrix}
\end{equation}

观测一维高斯函数可以发现,通过控制$\sigma, \mu$便可以控制一维高斯分布的曲线形状和中心位置;三维高斯函数同样地可以通过控制协方差矩阵$\bm{\Sigma}, \bm{\mu}$来控制三维高斯分布的椭球形状及其椭球中心位置。只不过这里的$\bm{\Sigma} \in \mathbb{R}^{3\times 3}$ 涉及的维度较大,随机生成9个数填入协方差矩阵$\bm{\Sigma}$是不行的,每个三维高斯椭球都对应一个$\bm{\Sigma}$,但不是每个$\bm{\Sigma}$都能对应一个椭球,只有$\bm{\Sigma}$半正定才可以创造出一个高斯椭球。



\section{三维高斯函数协方差 $\bm{\Sigma}$}

\textbf{各向同性(isotropic)}的三维高斯分布的协方差矩阵$\bm{\Sigma}$是对角矩阵,而且对角线元素都相同,非对角位置均为0,即协方差矩阵可由一个标量乘以单位阵$\bm{I}$得到。各向同性的三维高斯分布的几何表现为概率密度函数的等高线为多维球体(而不是椭球体)。


\textbf{各向异性(anisotropic)}的三维高斯分布的协方差矩阵$\bm{\Sigma}$的对角线元素各不相同,其物理意义是指各个坐标轴方向上的方差各不相同。 各向异性(anisotropic)的三维高斯分布的协方差矩阵$\bm{\Sigma}$并不是只可以对角线上有值,非对角线上的值也可以存在,这代表了不同坐标轴方向上的相关性。例如,非对角线上的值为零时,表示各个方向之间是独立的,没有相关性;非对角线上的值不为零时,表示不同方向之间存在一定程度的相关性。


一种明显的方法是直接优化协方差矩阵$\bm{\Sigma}$以获得表示辐射场的三维高斯分布。 然而,协方差矩阵$\bm{\Sigma}$仅在半正定时才具有物理意义。为了优化所有参数,学习过程中使用梯度下降算法,但梯度下降不能轻易地被约束来产生有效矩阵(半正定),在学习过程中如果直接回归$\bm{\Sigma}$,很容易导致创建无效的协方差矩阵$\bm{\Sigma}$。为此,《3D Gaussian Splatting for Real-Time Radiance Field Rendering》选择了一种更直观、但具有同等表达能力的表示方式来进行优化。三维高斯的协方差矩阵$\bm{\Sigma}$类似于描述椭球体的配置。给定缩放矩阵$S$和旋转矩阵$R$,可以找到相应的$\bm{\Sigma}$:


























\end{document}
